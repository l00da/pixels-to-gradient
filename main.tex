\documentclass[12pt]{article}
\usepackage{fullpage,enumitem,amsmath,amssymb,graphicx}

\title{CSCI-GA-2271 Fall 2025 Assignment 0}
\author{Your Name -- \texttt{vtp2019@nyu.edu}}

\begin{document}


\maketitle

\noindent
\rule{\linewidth}{0.4pt}

\section*{Question 1} 
P (1 wins) = 1/5, P (1 fail) = 4/5 P (2 wins) = 1/4, P (2 fail) = 3/4

P (1 fail, 2 fail) = (1-p1) (1-p2) = (4/5) (3/4) = 3/5 

P(1 wins) = P(1 wins) + P(1 fail, 2 fail, 1 wins) + P(1 fail, 2 fail , 1 fail, 2 fail, 1 wins) ..

P(1 wins) = 1/5 + P(1 fail, 2 fail) * (1/5) + P(1 fail, 2 fail)^2 * (1/5) .. = (1/5) * $$\frac{1}{1-P(1 fail, 2 fail)}$$ 

$$P(1 wins) = \frac{1}{5} + \frac{1}{1-3/5} = 0.5$$

because of geometric series, to simplify 1 + P(1 fail, 2 fail) + P(1 fail, 2 fail)^2 + P(1 fail, 2 fail)^3 ..

\vspace{10cm}

\newpage

\section*{Question 2} 
P(COVID) = 0.01, P(no COVID) = 0.99, P(+|COVID) =0.9, P(-|COVID)=0.1

$$P(COVID|+) = \frac{P(COVID\cap +)}{P(+)}$$

Expand numerator, $$P(COVID\cap+) = P(COVID)*P(+|COVID) = 0.01*0.9 =0.009$$

Denominator 0.009+(0.10)(0.99) = 0.009+0.099 = 0.108
$$\frac{0.009}{0.108} \approx 0.0833$$
\vspace{10cm}

\newpage

\section*{Question 3} 
Valid iff it maintains non-negativity and the \int_{-\infty}^{\infty}\frac{1}{1+x}f(x)dx = 1

so 

i) f(x)=0 for $x < 0$  and $f(x)=\frac{1}{1+x} > 0$ for $x \geq 0$ 

\vspace{1cm}

ii) \int_{-\infty}^{\infty}\frac{1}{1+x}f(x)dx = \int_{0}^{\infty}\frac{1}{1+x}dx = [ln(1+x)]_0^\infty 

\vspace{0.5cm}

$ln(1+0) = 0$

$ln(1+100,000) \approx 11.513$

$ln(1+1,000,000) \approx 13.8155$

\vspace{0.5cm}

$f(x)$ is not a valid PDF since \int_{-\infty}^{\infty}\frac{1}{1+x}f(x)dx$ diverges as x grows larger.

\vspace{10cm}

\newpage

\section*{Question 4} 

$P(X + Y \leq C) = \int\int_{x+y\leq c} f_x(x)f_y(y)dxdy,      X \perp Y $
\vspace{0.5cm}

$P(X + Y \leq C) = \int_0^1\int_0^{1-x} f_X(x)f_Y(y)dxdy$

$= \int_0^1\int_0^{1-x} (2x)(2y) dxdy$

$= \int_0^1 (4x)[\frac{y^2}{2}]^{1-x}_0dx$

$= \int_0^1 (2x(1-x)^2 dx$

$= \int_0^1 (2x-4x^2+2x^3) dx$

$= [x^3-\frac{4}{3}x^3 + \frac{1}{2}x^4]^1_0 = 1 - \frac{4}{3} + \frac{1}{2} = \frac{1}{6}$
\vspace{0.5cm}

$P(X + Y \leq C) = \frac{1}{6}$


\vspace{10cm}
\newpage

\section*{Question 5} 

X \sim \text{Unif}(0,1), 
\qquad Y = e^X.


The expectation of $Y$ is

$\mathbb{E}[Y] = \mathbb{E}[e^X] = \int_0^1 e^x \cdot f_X(x)\, dx$

Since $X \sim \text{Unif}(0,1)$, the pdf is

$f_X(x) = 1 \quad \text{for } 0 < x < 1$

continued

$\mathbb{E}[Y] = \int_0^1e^xdx$


\int_0^1e^xdx$ = e^x

e^1 - e^0 = e-1

\vspace{10cm}

\newpage

\section*{Question 6} 
We know that $X_i \sim \text{Poisson}(\lambda = 5), ; i=1,\dots,125.$

So the sum $S = \sum_{i=1}^{125} X_i \sim \text{Poisson}(125 \cdot 5) = \text{Poisson}(625).$

$\bar{X} = \frac{S}{125},$ and the question is about $\mathbb{P}(\bar{X} < 5.5).$

Since $125 \cdot 5.5 = 687.5$, this means
$\mathbb{P}(\bar{X} < 5.5) = \mathbb{P}(S \leq 687).$

$\mathbb{P}(S \leq 687) = e^{-625}\sum_{k=0}^{687} \frac{625^k}{k!},$
but that is not computable by hand.

CLT 

For $S$,
$\mu = 625, \quad \sigma^2 = 625, \quad \sigma = \sqrt{625} = 25.$

So we can do a normal approx:
$Z = \frac{687.5 - 625}{25} = \frac{62.5}{25} = 2.5.$

Therefore
$\mathbb{P}(S \leq 687) \approx \Phi(2.5) \approx 0.9938.$

So the answer is 
$\mathbb{P}(\bar{X} < 5.5) \approx 0.994.$

\vspace{10cm}

\newpage

\section*{Question 7} 

We have the recurrence 

$X_n = f(W_n, X_{n-1}), ; n=1,\ldots,p,$

and the error

$E = |c - X_p|^2 = (c - X_p)^\top (c - X_p).$

First,

$\frac{\partial E}{\partial X_p} = 2(X_p - c).$

Then by chain rule (back through the recurrence):

$\frac{\partial E}{\partial X_{p-1}} = \frac{\partial E}{\partial X_p},\frac{\partial X_p}{\partial X_{p-1}},$


$\frac{\partial E}{\partial X_{p-2}} = \frac{\partial E}{\partial X_{p-1}},\frac{\partial X_{p-1}}{\partial X_{p-2}},$


and so on, until we reach


$\frac{\partial E}{\partial X_0} = \frac{\partial E}{\partial X_p}\cdot \frac{\partial X_p}{\partial X_{p-1}}\cdot \frac{\partial X_{p-1}}{\partial X_{p-2}}\cdots\frac{\partial X_1}{\partial X_0}.$

Finally, substituting the form of each $X_n$:


$\frac{\partial E}{\partial X_0} = 2(X_p - c),\frac{\partial f(W_p, X_{p-1})}{\partial X_{p-1}},\frac{\partial f(W_{p-1}, X_{p-2})}{\partial X_{p-2}}\cdots\frac{\partial f(W_1, X_0)}{\partial X_0}.$
\vspace{10cm}

\newpage

\section*{Question 8} 
$Ax, A^Tx, x^TA$

\vspace{0.5cm}
Ax 

2(2) + 6(3) + 7(4) = 4+18+28 = 50

3(2) + 1(3) + 2(4) = 6+3+8 = 17

5(2) + 3(3) + 4(4) = 10+9+16 = 35

$Ax = \begin{bmatrix}
    50 \\ 17 \\ 35
\end{bmatrix}


$
$A^Tx$

$
A^T = \begin{bmatrix}
    2 & 3 & 5\\6 & 1 & 3 \\ 7&2&4
\end{bmatrix}$

\vspace{0.5cm}
$A^Tx $

2(2) + 3(3) + 5(4) = 4+9+20 = 33

6(2) + 1(3) + 3(4) = 12+3+12 = 27

2(7) + 2(3) + 4(4) = 14 + 6 + 16 = 36

$\begin{bmatrix}
    33  \\ 27 \\ 36
\end{bmatrix}
$

\vspace{0.5cm}
$x^TA $
$$
\begin{bmatrix}
    2 & 3 & 4
\end{bmatrix}
$$
2(2) + 3(3) + 4(5) = 4+9+20 = 33

2(6) + 3(1) + 4(3) = 12+3+12 = 27

2(7) + 3(2) + 4(4) = 14+6+16 = 36

$x^TA = \begin{bmatrix}
    33 &27 &36
\end{bmatrix}$

\vspace{10cm}

\newpage

\section*{Question 9} 

$a=6, b=2, c=3, d=3, e=1, f=1, g=10, h=3, i=4$

$det A = 6(1*4 -1*3) - 2(3*4-1*10) + 3(3*3-1*10) = 6(1) - 2(2) + 3(-1) = -1 != 0$

$A$ is invertible

I gave an LLM a snapshot of my work prompting it with the following: "Can you make this in latex"

\[
\begin{aligned}
C_{11} &= +\,\det\!\begin{bmatrix} 1 & 1 \\ 3 & 4 \end{bmatrix} 
       = 4 - 3 = 1, \\[6pt]
C_{12} &= -\,\det\!\begin{bmatrix} 3 & 1 \\ 10 & 4 \end{bmatrix} 
       = -(12 - 10) = -2, \\[6pt]
C_{13} &= +\,\det\!\begin{bmatrix} 3 & 1 \\ 10 & 3 \end{bmatrix} 
       = 9 - 10 = -1, \\[12pt]
C_{21} &= -\,\det\!\begin{bmatrix} 2 & 3 \\ 3 & 4 \end{bmatrix} 
       = -(8 - 9) = +1, \\[6pt]
C_{22} &= +\,\det\!\begin{bmatrix} 6 & 3 \\ 10 & 4 \end{bmatrix} 
       = 24 - 30 = -6, \\[6pt]
C_{23} &= -\,\det\!\begin{bmatrix} 6 & 2 \\ 10 & 3 \end{bmatrix} 
       = -(18 - 20) = +2, \\[12pt]
C_{31} &= +\,\det\!\begin{bmatrix} 2 & 3 \\ 1 & 1 \end{bmatrix} 
       = 2 - 3 = -1, \\[6pt]
C_{32} &= -\,\det\!\begin{bmatrix} 6 & 3 \\ 3 & 1 \end{bmatrix} 
       = -(6 - 9) = +3, \\[6pt]
C_{33} &= +\,\det\!\begin{bmatrix} 6 & 2 \\ 3 & 1 \end{bmatrix} 
       = 6 - 6 = 0.
\end{aligned}
\]

\[
C =
\begin{bmatrix}
1 & -2 & -1 \\
1 & -6 & 2 \\
-1 & 3 & 0
\end{bmatrix},
\qquad
\operatorname{adj}(A) = C^{\top} =
\begin{bmatrix}
1 & 1 & -1 \\
-2 & -6 & 3 \\
-1 & 2 & 0
\end{bmatrix}.
\]

Divide by the adj A/det A:

$
\begin{bmatrix}
-1 & -1 & 1 \\
2 & 6 & -3 \\
1 & -2 & 0
\end{bmatrix}.
\]




b)

\[
a=1,\; b=2,\; c=3,\; d=0,\; e=2,\; f=2,\; g=1,\; h=4,\; i=5
\]

\[
\det(B) 
= 1(2\cdot5 - 2\cdot4) \;-\; 2(0\cdot5 - 2\cdot1) \;+\; 3(0\cdot4 - 2\cdot1)
= (10 - 8) - 2(0 - 2) + 3(0 - 2)
= 2 + 4 - 6 = 0
\]

Therefore, B is not inversible



\vspace{10cm}
\newpage

\section*{Question 10} 

Eigenvectors are vectors, which represent the direction and  can be scaled by a scalar value $\lambda$, where the eigenvector's direction remains unchanged.  Eigenvalues are features of a matrix, which can also prove if a matrix is unequal or not.

$\det\!\left(\lambda \begin{bmatrix} 1 & 0 & 0 & \\ 0 & 1 & 0 & \\ 0 & 0 & 1 \end{bmatrix} - \begin{bmatrix} 1 & 0 & -1 & \\ 1 & 0 & 0 & \\ -2 & 2 & 1 \end{bmatrix}\right)$

\begin{bmatrix} \lambda-1 & 0 & 1 & \\ -1 & \lambda-0 & 0 & \\ 2 & -2 & \lambda-1 \end{bmatrix}


1st term$(\lambda-1)(\lambda(\lambda-1)) - (0_(-2)) = (\lambda-1)(\lambda(\lambda-1)) = (\lambda-1)(\lambda^2-\lambda)$

2nd term b=0

3rd term $c(dh-eg) = 1((-1)(-2) -(\lambda)(2)) = 2-2\lambda$




\lambda^3-2\lambda^2+\lambda+(2-2\lambda)

=\lambda^3-2\lambda^2-\lambda+2

=(\lambda-2)(\lambda-1)(\lambda+1)

eigenvalues \lambda\in {2, 1, -1}

\vspace{10cm}

\end{document}
